\section{Code Legibility and Output}

This is not strictly part of the report but is a requirement for the
final hand-in.

\begin{itemize}
\item Each method should start wide a brief description of its
  function.

\item Use indentation to display the structure within a method.

\item Comments should be used extensively. They are best used to
  describe logical blocks of code rather than individual
  statements. Line-by-line comments have the drawbacks of not
  providing any overview and of decreasing readability.

\item Meaningful identifiers should be chosen.

\item Output should be pleasingly formatted and easy to read.
\end{itemize}

You do, of course, have the option to call in any of your
favourite packages for setting maths, graphics, computer listings,
etc.

\begin{thebibliography}{9}

\bibitem[Kopka and Daly(2004)]{KopkaDaly}
Kopka, H. and Daly, P.W.  (2004) \textit{A Guide to \LaTeXe:
Document Preparation for Beginners and Advanced Users} (4th~edn).
Addison-Wesley.

\bibitem[Lamport(1994)]{Lamport}
Lamport L. (1994) \textit{\LaTeX: A Document Preparation System}
(2nd~edn). Addison-Wesley.

\bibitem[Mittelbach and Goossens(2004)]{Companion}
Mittelbach, F. and Goossens, M., (2004) \textit{The \LaTeX\
Companion} (2nd~edn). Addison-Wesley.

\end{thebibliography}