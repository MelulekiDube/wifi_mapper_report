\section{Introduction}
The capstone project is done as the culmination of your three year
study of Computer Science. It is the development of a real application
that draws on all your knowledge of the field gained in the course of
your training in the subject.

The project should be written up as a professional software
engineering design and development project. We expect a report of
about 3500-4000 words, written single spaced, with a font size of at
least 11 pts. Use at least a 2.5 cm margin on all sides of the
pages. Please use ``styles'' for formatting if you are using a word
processing package or use \LaTeX. We mean you to use styles for
everything, not just headings and lists but also for a different
font. So use emphasis for \emph{italics} and do \emph{not} use a bold
face. No blank lines between paragraphs except to get figures and
their captions to position properly.

Depending on how many diagrams you use (more is better) the report
will be between 7 and 10 pages long. Note that while word users will
struggle with numbered headings and lists, \LaTeX\/ has its own ideas
about where ``floats'' (like tables and figures) will go, as usual
search the internet for advice (e.g., search for ``\LaTeX quick
guide'' and look at
http://www.andy-roberts.net/writing/latex/floats\_figures\_captions). Don't
worry, word's specialty is loosing your figures in some between-page
limbo; \LaTeX\/ will not loose them, just place them way after the
spot where you want them.  This document shows the format we expect
and you can use it as a template.  Your appendices (e.g., user manual,
test results, which are needed) are not included in these limits.

You must had-in an Adobe Acrobat file for your report (i.e., pdf
file). Not word, latex source, but \emph{PDF}!
