\section{Introduction}
\label{ss:introduction}
The project was concerned with the creation of a Wi-Fi signal strength mapper of the UCT eduroam internet across the upper Campus.
The stakeholders who were involved in the project included:
\begin{itemize}
	\item \textbf{Product Owner/Client:} This is the person in-charge of the project and for whom the project is being built for. In this case the client came with the project Idea which we then implemented in a way we saw fit. \textit{The project owner in this instance is Dr. Josiah Chavula}
	\item \textbf{Basic users of the application} This is everyone who will download the application on their phone are going to using the phone to view the Wi-Fi strength of the eduroam networks in different areas on on UCT upper campus. 
	\paragraph{}The basic end user of the application will also be responsible for collecting the data that will be used in the future to determine the average strength of a area. This has to be done with their permission.
	\item \textbf{Advanced users/ICTS department} These users will mainly use the application to monitor the Wi-Fi of the network across campus to act on the data.
\end{itemize} The Wi-Fi Mapper project consisted of two main deliverables,android application for the basic users together with a web application for advanced users.
\paragraph{} 
The purpose of the application was to provide a way for the students to know which places on campus have the best Wi-Fi strength so that they can go to those areas to use the Wi-Fi for applications that require strong Wi-Fi signals to operate appropriately. The problems together with the opportunities that arise from this projects are:
\begin{itemize}
	\item \textbf{Problems:}
	\begin{itemize}
		\item One has to travel to a further distance to an access point with a strong wifi signal
		\item New users of eduroam service do not have enough experience to know where strong
		signals can be found on campus
	\end{itemize}
	\item  \textbf{Opportunities:}
	\begin{itemize}
		\item  Give a wider range of locations with a strong signal strength
		\item Maintenance easier for ICTS
	\end{itemize}
\end{itemize} 
The Wi-Fi mapper then serves as the platform to provide information and overview of the signal areas of the campus for the stake holders mentioned above.
\subsection*{Software engineering methods used}
The project was developed in an iterative manner. Therefore the methodology used for this project was Agile. Agile development methods include developing the application in both iterative and incremental steps. Each end of an iteration is met with a milestone set upon by the client and the development team. The reason for going with this approach  was so as to keep the client in the development process so as to not incorporate the change in requirements that the client might have and also to ensure that the client was happy with the progress at each development stage. At the end of each iteration the client was consulted to ensure the application being developed was what the client had envisioned,that way instant Client feedback was given back in the case things were not going as planned. The model used in the project was the evolutionary prototyping method where the application was built in iterations of prototypes. These prototypes where shown to the user at the end of each iteration.

