\section{Conclusion}
The goal of the project  was to develop a WiFi Mapper, an app that allows users to view the quality of Eduroam wifi in different WLAN zones at upper campus.

\paragraph{}Our team set out to create a WiFi mapper that would meet the specified user requirements. The features implemented on the app are such that the user can easily determine WLAN zones with the strongest wifi strength.

\paragraph{}This was achieved through the use of different colors on each WLAN zone to depict the strength of the wifi. WLAN zones with different wifi strengths have different colors.The data used to determine the color of the WLAN zone was collected from WiFi routers using the app running in the background.

\paragraph{}Another requirement was that the WiFi mapper should be able to display data on a mobile client and on a web client. Our Wifi mapper managed to do this by having an app running on android devices and on the web. However only the native app is used to collect data from routers.

\paragraph{}The web app only takes data from the database and displays it to the user in the same way the native app does. This allows consistency across the two apps so that the user can have the same experience on both apps. To further meet the requirements that involve user experience, WiFi mapper allows users to zoom and pan on the map as they deem necessary. 

\paragraph{}For advanced like ICTS members of stuff the app allows them to generate a report that shows the perfomance Eduroam accross campus. The report gives numbers of datapoints and number of datapoints collected to the user. It also displays interactive graphs that the user can interact with as the data is collected by the users. The developed app therefore meets most of the requirements that were initially put accross by the client for the users.

